\documentclass{article}
\usepackage{OzanerTeX}

\begin{document}
\title{\Ozaner\TeX{} Documentation}
\author{Ozaner Hansha}
\date{}
\maketitle
\setcounter{section}{-1}

\section*{What is \Ozanerbf\TeX{}?}
The eponymous \Ozaner\TeX{} is a \TeX{} package of miscellaneous commands, preformatting, and default package inclusions. Rather than append these definitions at the top of all my \LaTeX{} files, I decided to factor them out as a package.

This document will serve as documentation, both in the form of examples and rationale, of this package. It also serves as a sort of unit test because, if it compiles, the package probably works.

\tableofcontents

\newpage{}

\section{Preformatting}
\subsection{Margins}
Via the \texttt{geometry} package, the default margin size has been changed to 1.5in.\ For reference, the article document class uses a margin size of 1.875in.

\subsection{Title starting height}
Using the \texttt{titling} package, the starting height of the title has been set back by 7em.

\subsection{Hyper-reference styling}
Instead of colored boxes, hyper-references use colored text instead:
\begin{itemize}
  \item Internal link: \textcolor{blue}{blue}
  \item File link: \textcolor{magenta}{magenta}
  \item URL: \textcolor{cyan}{cyan}
\end{itemize}

\section{Base packages}
The \Ozaner\TeX{} package includes several \texttt{\textbackslash usepackage} declarations. These are packages that I use commonly enough to simply include in every document. These packages are, in order of inclusion, listed below:

\noindent\\\textit{Note}: Nested bullets denote packages that are included as dependencies of the parent bullet. These are only listed if I use them independently of their parent package.

\begin{itemize}
  \item \href{https://ctan.org/pkg/titling}{\texttt{fontenc}}: properly renders certain special characters in text-mode.
  \item \href{https://ctan.org/pkg/geometry}{\texttt{geometry}}: used to adjust the document margins.
  \item \href{https://ctan.org/pkg/titling}{\texttt{titling}}: used to push back the title starting height.
  \item \href{https://ctan.org/pkg/hyperref}{\texttt{hyperref}}: adds commands for hyper-referencing, and adds hyper-references to the table of contents.
  \item \href{https://ctan.org/pkg/amsfonts}{\texttt{amssymb}}: Adds many useful mathematical symbols (e.g.\ blackboard bold letters ($\mathbb{R}$), arrows ($\curvearrowright$), inequalities ($\ngeq$), etc.).
  \item \href{https://ctan.org/pkg/physics}{\texttt{physics}}: adds a whole bunch of common functions (i.e.\ trig, logs, exp, matrix ops) as well as commands for typesetting matrices, derivatives, and vectors. Crucially, can now use $\vb v$ (\texttt{\textbackslash vb v}) instead of $\vec v$ (\texttt{\textbackslash vec v}) for vectors.
  \begin{itemize}
    \item \href{https://ctan.org/pkg/amsmath}{\texttt{amsmath}}: A general math package that adds many misc.\ features including: equation alignment, matrix environments, fraction variants, extensible arrows, creating operators, and substacks.
  \end{itemize}
\end{itemize}

\section{Textual Commands}
\subsection{\texttt{\textbackslash Ozaner} \& \texttt{\textbackslash Ozanerbf}}
\texttt{\textbackslash Ozaner\{\}} $\longrightarrow$ \Ozaner{}\\
\texttt{\textbackslash Ozanerbf\{\}} $\longrightarrow$ \Ozanerbf{}

\noindent\\The stylized version of my name I use in various places, including this package. The bolded version (i.e.\ \texttt{\textbackslash Ozanerbf}) is necessary as \texttt{\textbackslash textbf} won't boldface embedded math symbols.

\noindent\\\textit{Note}: to typeset \Ozaner\TeX{} simply append \texttt{\textbackslash TeX} to the command, i.e.\ \texttt{\textbackslash Ozaner\textbackslash TeX\{\}}.

\section{Calculus Commands}
\subsection{\texttt{\textbackslash evalb}}
\texttt{\textbackslash evalb\{f(x)\}\{a\}\{b\}} $\longrightarrow\evalb{f(x)}{a}{b}$

\noindent\\Intended to denote the result of a definite integral after integration, but before evaluation:
$$\int_0^5 2x\diff x = \evalb{x^2}{0}{5} = 5^2-0^2$$


\subsection{\texttt{\textbackslash diff}}
\texttt{\textbackslash diff x} $\longrightarrow\diff x$

\noindent\\The \texttt{\textbackslash diff} is simply the \texttt{\textbackslash dd} command, from the \texttt{physics} package, prepended with a space. It, along with a variable following it, are intended to be used as the differential in an integral:
$$\int x\diff x = x^2+C$$

Without the prepended space, i.e.\ using \texttt{\textbackslash dd}, the differential is too close to the integrand:
$$\int x\dd x = x^2+C$$

\section{Set Theory Commands}
\subsection{\texttt{\textbackslash pset}}
\texttt{\textbackslash pset X} $\longrightarrow\pset X$

\noindent\\Used to denote the powerset of some set $X$. For example:
$$\pset{\{1,2\}}=\{\varnothing,\{1\},\{2\},\{1,2\}\}$$

\subsection{\texttt{\textbackslash N, \textbackslash Z, \textbackslash Q, \textbackslash R, \textbackslash C, \textbackslash H}}
\texttt{\textbackslash N} $\longrightarrow\N$\quad
\texttt{\textbackslash R} $\longrightarrow\R$\\
\texttt{\textbackslash Z} $\longrightarrow\Z$\quad
\texttt{\textbackslash C} $\longrightarrow\C$\\
\texttt{\textbackslash Q} $\longrightarrow\Q$\quad
\texttt{\textbackslash H} $\longrightarrow\H$

\noindent\\These are simply shorthands for the sets of 6 common number systems. Note that the quaternions command (\texttt{\textbackslash H}) overrides a different, unimportant, command.

\end{document}